\documentclass[11pt,oneside]{amsart}

\title{Fragenkatalog zur Stochastischen Analysis (SS 2014)}

\usepackage[utf8]{inputenc}\usepackage[T1]{fontenc}
\usepackage{enumitem, amsfonts, amssymb, amsmath, latexsym, lmodern}
\usepackage[ngerman]{babel}
\newcommand{\pp}[1]{\phantom{}^{p}#1}
\newcommand{\bF}{\mathbb F}\newcommand{\cF}{\mathcal F}\newcommand{\fF}{\mathfrak F}\newcommand{\bfF}{\mathbf F}
\newcommand{\bH}{\mathbb H}\newcommand{\cH}{\mathcal H}\newcommand{\fH}{\mathfrak H}\newcommand{\bfH}{\mathbf H}
\newcommand{\bV}{\mathbb V}\newcommand{\cV}{\mathcal V}\newcommand{\fV}{\mathfrak V}\newcommand{\bfV}{\mathbf V}
\newcommand{\bO}{\mathbb O}\newcommand{\cO}{\mathcal O}\newcommand{\fO}{\mathfrak O}\newcommand{\bfO}{\mathbf O}
\newcommand{\bP}{\mathbb P}\newcommand{\cP}{\mathcal P}\newcommand{\fP}{\mathfrak P}\newcommand{\bfP}{\mathbf P}


\begin{document}
\maketitle

\begin{enumerate}
    \item Definieren Sie die Stochastische Basis und diskutieren Sie die
        üblichen Bedingungen.
    \item Definieren Sie einen stochastischen Prozess. Wann ist ein
        stochastischer Prozess adaptiert? Geben Sie die Definitionen und je ein
        Beispiel von einem c\`adl\`ag, c\`agl\`ad, und einem stetigen Prozess.
    \item Definieren Sie eine Stoppzeit. Geben Sie ein Beispiel einer Stoppzeit
        und einer Zufallsvariable die keine Stoppzeit ist. Definieren Sie die
        Stoppzeit-$\sigma$-Algebra $\cF_{T}$ bzgl.\ einer Stoppzeit $T$. Wie
        können die Ereignisse in $\cF_{T}$ beschrieben werden?
    \item Definieren Sie die optionale $\sigma$-Algebra $\cO$ und die
        vorhersehbare $\sigma$-Algebra $\cP$. Wie können diese
        $\sigma$-Algebren erzeugt werden? Ist die Brownsche Bewegung
        vorhersehbar?
    \item Definieren Sie eine lokalisierende Folge. Wann ist eine Klasse
        stochastischer Prozesse stabil unter Stoppen? Wie kann die Methode der
        Lokalisierung gewinnbringend eingesetzt werden? Geben Sie ein Beispiel
        eines lokal beschränkten Prozesses der nicht beschränkt ist. 

    \item Definieren Sie die Begriffe Martingal, Submartingal, Supermartingal.
        Geben Sie je ein Beispiel einen Martingals, und eines Sub- und
        Supermartigals die keine Martingale sind. Zitieren Sie die wichtigsten 
        Sätze der Martingaltheorie: Die Doob'schen Ungleichungen, Theoreme
        über optionales Stoppen und Martingal Konvergenzsätze.

    \item Was ist ein gleichmäßig integrierbarer Martingal? Wieso können solche
        Martingale als Martingale auf $[0,\infty ]$ angesehen werden?

    \item Was ist ein lokales Martingal? Ist ein Prozess der lokal ein lokales
        Martingal ist, ein lokales Martingal? Geben Sie ein Beispiel eines
        lokalen Martingals, der kein Martingal ist.

    \item Was ist eine vorhersehbare Stoppzeit? Was ist eine unerreichbare Stoppzeit? 
        Was ist der Zusammenhang zwischen diesen Begriffen? 

    \item Definieren Sie die vorhersehbare Projektion $\pp{X}$. Ist $\pp{X}$ eindeutig
        definiert? Was ist $\pp{X}$ wenn $X$ ein lokales Martingal ist?

    \item Definieren Sie die Klassen der wachsenden Prozesse $\cV^{+}$ und der
        Prozesse von endlicher Variation $\cV$. Wie können stochastische Integrale
        bezüglich dieser Prozesse definiert werden?

    \item Definieren Sie die Klassen der Semimartingale und der speziellen Semimartingale. 
        Welche Stabilitätseigenschaften können Sie nennen? Wie können Semimartingale und
        speziellen Semimartingale zerlegt werden?

    \item Wie können ein stochastisches Integral bzgl.\ eines Semimartingals
        definiert werden?  Diskutieren Sie die Konstruktion des stochastischen
        Integrals für einfache Prozesse und zeigen Sie wie Prozesse aus
        allgemeineren Klassen wie $\cH^{2}_{loc}$ mit Hilfe geeigneter
        Grenzübergänge integriert werden können. Welche Eigenschaften eines so
        konstruierten Integrals können Sie nennen?

    \item Zitieren Sie die It\^o Formel und diskutieren Sie warum diese fundamentale Bedeutung
        in der Stochastischen Analysis hat. Erläutern Sie wir man mit Hilfe der It\^o Formel 
        Doleans-Dade Exponential konstruiren kann. 

\end{enumerate}

\end{document}
