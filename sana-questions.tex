\documentclass[11pt,oneside]{amsart}

%%% packages

\usepackage[final]{graphicx}   %% display figures in draft mode
\usepackage{verbatim, color, amsfonts, amssymb, amsmath, latexsym, MnSymbol}   % lipsum
\usepackage{listings}\lstset{basicstyle=\ttfamily, breaklines=true}
\makeindex

\usepackage{hyperref}
\hypersetup{colorlinks=true, linkcolor=black, bookmarksopenlevel={2}, bookmarksopen = true}
%\hypersetup{pdfauthor={\theauthor}, pdftitle={\thetitle}}   %%  This is *memoir* specific

\usepackage{showkeys}
\usepackage[obeyDraft,backgroundcolor=white]{todonotes}

\usepackage{mathtools}%                  http://www.ctan.org/pkg/mathtools
%\usepackage[tableposition=top]{caption}% http://www.ctan.org/pkg/caption
%\usepackage{booktabs,dcolumn}%           http://www.ctan.org/pkg/dcolumn + http://www.ctan.org/pkg/booktabs


\usepackage{amsmath,amssymb,amsthm}

\theoremstyle{plain}
\newtheorem{theorem}{Theorem}
\newtheorem{acknowledgement}[theorem]{Acknowledgement}
\newtheorem{algorithm}[theorem]{Algorithm}
\newtheorem{axiom}[theorem]{Axiom}
\newtheorem{case}[theorem]{Case}
\newtheorem{claim}[theorem]{Claim}
\newtheorem{conclusion}[theorem]{Conclusion}
\newtheorem{conjecture}[theorem]{Conjecture}
\newtheorem{corollary}[theorem]{Corollary}
\newtheorem{criterion}[theorem]{Criterion}
\newtheorem{exercise}[theorem]{Exercise}
\newtheorem{lemma}[theorem]{Lemma}
\newtheorem{notation}[theorem]{Notation}
\newtheorem{problem}[theorem]{Problem}
\newtheorem{proposition}[theorem]{Proposition}
\newtheorem*{solution}{Solution}
\newtheorem{definition}[theorem]{Definition}
\newtheorem{condition}[theorem]{Condition}

\theoremstyle{remark}
\newtheorem{assumption}{Assumption}
\newtheorem{remark}[theorem]{Remark}
\newtheorem{example}[theorem]{Example}
\newtheorem{summary}[theorem]{Summary}

\newtheorem{questions}[theorem]{Questions}






% Problem collection commands
%\newcommand{\problem}[1]{\paragraph{#1}}
%\newcommand{\solution}{\paragraph*{Lösung.}}

% Annotation
\newcommand{\key}[1]{\textsl{#1}\index{#1}}
\newcommand{\starit}{$\bigstar$}
\newcommand{\Q}{{\textbf{Q: }\index{Questions}}}
\newcommand{\kw}{\emph{Keywords: }}
\newcommand{\src}{\emph{Source: }}
\renewcommand{\iff}{\Leftrightarrow}

% General
\newcommand{\R}{\mathbb{R}}
\newcommand{\C}{\mathbb{C}}
\newcommand{\N}{\mathbb{N}}
\newcommand{\ind}{1}

%Stochastics
\newcommand{\E}{\mathbb{E}}

% Logic
\newcommand{\impl}{\Rightarrow}
\DeclareMathOperator{\cov}{cov}
\DeclareMathOperator{\corr}{corr}
\DeclareMathOperator{\var}{var}\DeclareMathOperator{\Var}{Var}

\newcommand{\cond}{\;\middle\vert\;}


\newcommand{\bA}{\mathbb A}\newcommand{\cA}{\mathcal A}\newcommand{\fA}{\mathfrak A}\newcommand{\bfA}{\mathbf A}
\newcommand{\bB}{\mathbb B}\newcommand{\cB}{\mathcal B}\newcommand{\fB}{\mathfrak B}\newcommand{\bfB}{\mathbf B}
\newcommand{\bC}{\mathbb C}\newcommand{\cC}{\mathcal C}\newcommand{\fC}{\mathfrak C}\newcommand{\bfC}{\mathbf C}
\newcommand{\bD}{\mathbb D}\newcommand{\cD}{\mathcal D}\newcommand{\fD}{\mathfrak D}\newcommand{\bfD}{\mathbf D}
\newcommand{\bE}{\mathbb E}\newcommand{\cE}{\mathcal E}\newcommand{\fE}{\mathfrak E}\newcommand{\bfE}{\mathbf E}
\newcommand{\bF}{\mathbb F}\newcommand{\cF}{\mathcal F}\newcommand{\fF}{\mathfrak F}\newcommand{\bfF}{\mathbf F}
\newcommand{\bG}{\mathbb G}\newcommand{\cG}{\mathcal G}\newcommand{\fG}{\mathfrak G}\newcommand{\bfG}{\mathbf G}
\newcommand{\bH}{\mathbb H}\newcommand{\cH}{\mathcal H}\newcommand{\fH}{\mathfrak H}\newcommand{\bfH}{\mathbf H}
\newcommand{\bI}{\mathbb I}\newcommand{\cI}{\mathcal I}\newcommand{\fI}{\mathfrak I}\newcommand{\bfI}{\mathbf I}
\newcommand{\bJ}{\mathbb J}\newcommand{\cJ}{\mathcal J}\newcommand{\fJ}{\mathfrak J}\newcommand{\bfJ}{\mathbf J}
\newcommand{\bK}{\mathbb K}\newcommand{\cK}{\mathcal K}\newcommand{\fK}{\mathfrak K}\newcommand{\bfK}{\mathbf K}
\newcommand{\bL}{\mathbb L}\newcommand{\cL}{\mathcal L}\newcommand{\fL}{\mathfrak L}\newcommand{\bfL}{\mathbf L}
\newcommand{\bM}{\mathbb M}\newcommand{\cM}{\mathcal M}\newcommand{\fM}{\mathfrak M}\newcommand{\bfM}{\mathbf M}
\newcommand{\bN}{\mathbb N}\newcommand{\cN}{\mathcal N}\newcommand{\fN}{\mathfrak N}\newcommand{\bfN}{\mathbf N}
\newcommand{\bO}{\mathbb O}\newcommand{\cO}{\mathcal O}\newcommand{\fO}{\mathfrak O}\newcommand{\bfO}{\mathbf O}
\newcommand{\bP}{\mathbb P}\newcommand{\cP}{\mathcal P}\newcommand{\fP}{\mathfrak P}\newcommand{\bfP}{\mathbf P}
\newcommand{\bQ}{\mathbb Q}\newcommand{\cQ}{\mathcal Q}\newcommand{\fQ}{\mathfrak Q}\newcommand{\bfQ}{\mathbf Q}
\newcommand{\bR}{\mathbb R}\newcommand{\cR}{\mathcal R}\newcommand{\fR}{\mathfrak R}\newcommand{\bfR}{\mathbf R}
\newcommand{\bS}{\mathbb S}\newcommand{\cS}{\mathcal S}\newcommand{\fS}{\mathfrak S}\newcommand{\bfS}{\mathbf S}
\newcommand{\bT}{\mathbb T}\newcommand{\cT}{\mathcal T}\newcommand{\fT}{\mathfrak T}\newcommand{\bfT}{\mathbf T}
\newcommand{\bU}{\mathbb U}\newcommand{\cU}{\mathcal U}\newcommand{\fU}{\mathfrak U}\newcommand{\bfU}{\mathbf U}
\newcommand{\bW}{\mathbb W}\newcommand{\cW}{\mathcal W}\newcommand{\fW}{\mathfrak W}\newcommand{\bfW}{\mathbf W}
\newcommand{\bV}{\mathbb V}\newcommand{\cV}{\mathcal V}\newcommand{\fV}{\mathfrak V}\newcommand{\bfV}{\mathbf V}
\newcommand{\bX}{\mathbb X}\newcommand{\cX}{\mathcal X}\newcommand{\fX}{\mathfrak X}\newcommand{\bfX}{\mathbf X}
\newcommand{\bY}{\mathbb Y}\newcommand{\cY}{\mathcal Y}\newcommand{\fY}{\mathfrak Y}\newcommand{\bfY}{\mathbf Y}
\newcommand{\bZ}{\mathbb Z}\newcommand{\cZ}{\mathcal Z}\newcommand{\fZ}{\mathfrak Z}\newcommand{\bfZ}{\mathbf Z}

\title{Fragenkatalog zur Stochastischen Analysis (SS 2014)}
\author{Julian Wergieluk}

\usepackage[utf8]{inputenc}\usepackage[T1]{fontenc}
\usepackage{enumitem}
\usepackage{lmodern}
\usepackage[ngerman]{babel}
\newcommand{\pp}[1]{\phantom{}^{p}#1}


\begin{document}
\maketitle

\begin{enumerate}
    \item Definieren Sie die Stochastische Basis und diskutieren Sie die
        üblichen Voraussetzungen.
    \item Definieren Sie einen stochastischen Prozess. Wann ist ein
        stochastischer Prozess adaptiert?
    \item Geben Sie die Definitionen und je ein Beispiel von einem c\`adl\`ag,
        c\`agl\`ad, und einem stetigen Prozess.
    \item Definieren Sie eine Stoppzeit. Geben Sie ein Beispiel einer Stoppzeit
        und einer Zufallsvariable die keine Stoppzeit ist.
    \item Definieren Sie die Stoppzeit-$\sigma$-Algebra $\bF_{T}$ bzgl.\ einer 
        Stoppzeit $T$. Wie können die Ereignisse in $\cF_{T}$ beschrieben werden?
    \item Definieren Sie die optionale $\sigma$-Algebra $\cO$ und die
        vorhersehbare $\sigma$-Algebra $\cP$. Wie können diese
        $\sigma$-Algebren erzeugt werden? Ist die Brownsche Bewegung
        vorhersehbar?
    \item Definieren Sie eine lokalisierende Folge. Wann ist eine Klasse stochastischer
        Prozesse stabil unter Stoppen. Geben Sie ein Beispiel eines lokal
        beschränkten Prozesses der nicht beschränkt ist. 
    \item Was ist eine dünne Menge? Wie kann eine dünne Menge mit Hilfe der
        Graphen geeignet gewählten Stoppzeiten dargestellt werden?

    \item Definieren Sie die Begriffe Martingal, Submartingal, Supermartingal.
        Geben Sie je ein Beispiel einen Martingals, und eines Sub- und
        Supermartigals die keine Martingale sind.

    \item Martingale Convergence Theorem.
    \item Doob's Optional Sampling Theorems.
    \item Zitieren Sie die Doob'sche Ungleichung.

    \item Was ist ein gleichmäßig integrierbarer Martingal? Wieso können solche
        Martingale als Martingale auf $[0,\infty]$ angesehen werden?
    \item Was ist ein lokales Martingal? Ist ein Prozess der lokal ein lokales
        Martingal ist, ein lokales Martingal? Geben Sie ein Beispiel eines
        lokalen Martingals, der kein Martingal ist.

    \item Was ist eine vorhersehbare Stoppzeit? Was ist eine unerreichbare Stoppzeit? 
        Was ist der Zusammenhang zwischen diesen Begriffen? 

    \item Definieren Sie die vorhersehbare Projektion $\pp{X}$. Ist $\pp{X}$ eindeutig
        definiert? Was ist $\pp{X}$ wenn $X$ ein lokales Martingal ist?

    \item Definieren Sie die Klassen der wachsenden Prozesse $\cV^{+}$ und der
        Prozesse von endlicher Variation $\cV$. Wie können stochastische Integrale
        bezüglich dieser Prozesse definiert werden?
    \item Definieren Sie den vorhersehbaren Kompensator.
    \item Definieren Sie die Klassen der Semimartingale und der speziellen Semimartingale. 
    \item 
    \item 

\end{enumerate}



\subsection{Introduction to Semimartingales}

\begin{enumerate}[resume]
    \item Simple predictable process.
    \item Total semimartingale. Semimartingale.
    \item Semimartingales form a vector space.
    \item What parts of the definition of a (total) semimartingale depend on measure? 
        Measure manipulations.
    \item What parts of the definition of a semimartingale depend on filtration? 
    \item Shrinking of a filtration (Sticker's Theorem). Jacod's countable expansion.
    \item Local semimartingale is a semimartingale.
    \item Examples of semimartingales: Finite variation process. $L^2$ martingale. Brownian Motion. Levy process.
    \item ucp convergence.
    \item Space $\mathcal S$ of simple predictable processes is dense in $\mathbb L$ (caglad processes) under the ucp topology.
    \item Stochastic integral $J_X: \mathcal{S}_{ucp} \to \mathbb{D}_{ucp}$ is continuous.
    \item Extension of the domain of the stochastic integral to $\mathbb L$.
    \item Random partition tending to unity.
    \item Quadratic variation. Quadratic covariation. Existence. Polarization identity.
    \item Quadratic pure jump semimartingale. Example: FV process.
    \item It\^o-formula for semimartingales. $n$-dimensional It\^o formula. Compensator of $e^{N_t}$.
    \item Stochastic exponential.
\end{enumerate}












\end{document}
