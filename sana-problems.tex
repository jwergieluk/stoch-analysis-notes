
\section{Measurability of stochastic processes}

\problem{Progressive $\sigma$-algebra.}
Let $(\Omega, \cA, \cF, P)$ be a probability space satisfying the usual hypothesis. 
The following holds: 
\begin{enumerate}
    \item An adapted c\`adl\`ag or c\`agl\`ad process is progressively
        measurable. This implies that the optional $\sigma$-algebra $\cO$ and 
        the predictable $\sigma$-algebra $\cP$ are contained in the 
        progressive $\sigma$-algebra $\cR$. 

    \item Conversely, a measurable and adapted process is progressively
        measurable.

    \item $\cP \subset \cO$. 

    \item $\cR \subset\cM$, where $\cM$ is the measurable $\sigma$-algebra on
        $\Omega\times \R_{+}$. 
\end{enumerate}

\problem{Filtations and continuity of paths.} Let $X$ be a stochastic process
on a probability space $(\Omega, \cA, P)$ and $\cF^{X}=(\cF^{X}_{t})_{t\geq 0}$
be the natural filtration of $X$. Moreover, $A$ is the event that $X$ is
continuous on $[0, t_0)$. Then following holds:
\begin{enumerate}
    \item If $X$ has c\`adl\`ag paths, i.e. every path of $X$ is a c\`adl\`ag
        function, then $A\in \cF^{X}_{t_0}$. 

    \item If $X$ is c\`adl\`ag, i.e. $X$ has c\`adl\`ag paths a.s., then $A$
        may fail to be in $\cF^{X}_{t_0}$.

    \item If $X$ is c\`adl\`ag, and $(\cF_{t})_{t\geq 0}$ is a filtration
        containing $\cF^{X}$ such that $\cF_{0}$ contains all $P$-null events
        in $\cA$, then $A\in \cF_{t_0}$.

    \item If $X$ has c\`agl\`ad paths and is adapted to a right-continuous
        filtration $(\cF_{t})_{t\geq 0}$, then $A\in \cF_{t_{0}}$. 
\end{enumerate}


\section{Stopping times}

\problem{Stopping times. One-liners.} Let $T$ and $S$ be stopping times. 
\begin{enumerate}
    \item $T + t_0$ for a $t_0\geq 0$ is a stopping time.
    \item A fixed time $T \equiv t_0>0$ is a stopping time and we have $\cF_T = \cF_{t_0}$.
\end{enumerate}

\solution
\begin{enumerate}
    \item \begin{align*}
            \cF_T &=& 
            \left\{ A\in\bF : A \cap \left\{ t_0 \leq t \right\}\in\cF_t \ \forall t\geq 0 \right\} \\
            &=& \left\{ A\in\bF : A\in\cF_t \ \forall t\geq t_0 \right\} = \cF_{t_0}.
        \end{align*}
\end{enumerate}

\problem{Stopping times and optional times.} An optional time $T$ with respect
to the filtration $\bF$ is a stopping time, if $\bF$ is right continuous, i.e.\
$\cF_t=\cF_{t+}$ for all $t\geq 0$.

\solution  Betrachte die Darstellung
\begin{eqnarray}
    \left\{ T \leq t \right\} = \bigcap_{\varepsilon\in\bQ^{+}} \left\{ T > t+\varepsilon \right\}
    = \lim_{\varepsilon\to 0, \varepsilon\in\bQ} \left\{ T > t+\varepsilon \right\}.
\end{eqnarray}
Der Schnitt ist als Mengengrenzwert zu verstehen, weil die Mengen $\left\{ T >
t+\varepsilon \right\}$ ineinandergeschachtelt sind. $\left\{ T \leq t \right\}\in \bF_{t+}$
genau dann, wenn $\left\{ T \leq t \right\}\in \bF_{t+\varepsilon^{*}}$ $\forall \varepsilon^{*}>0$.
Es gilt aber für ein fixes $\varepsilon^{*}$
\begin{equation}
    \left\{ T \leq t \right\}\ = 
    \bigcap_{\varepsilon\in\bQ^{+}, \varepsilon\leq \varepsilon^{*}} \left\{ T > t+\varepsilon \right\} 
    \in\bF_{t+\varepsilon^{*}},
\end{equation}
denn es ein abzählbarer Schnitt der Mengen $\left\{ T> t+\varepsilon
\right\}\in\bF_{t+\varepsilon^{*}}$ ist. Behauptung gilt wegen $\cF_t=\cF_{t+}$ $\forall t$.


