
\section{Measurability of stochastic processes}

\problem{Progressive measurability. Definition and basic properties.} 
An adapted c\`adl\`ag or c\`agl\`ad process is
progressively measurable. 

\problem{Progressive measurability. Stopping times.} Let $X$ be a
progressively measurable process and $T$ a stopping time on a filtered
probability space $(\Omega, \cA, \bF, P)$. Then $X^{T}$ is progressively
measurable with respect to the filtration $\bF^{T} = (\cF_{T \wedge t})_{t\geq
0}$. 


\problem{Progressive $\sigma$-algebra.}
Let $(\Omega, \cA, \cF, P)$ be a probability space satisfying the usual hypothesis. 
The following holds: 
\begin{enumerate}
    \item An adapted c\`adl\`ag or c\`agl\`ad process is progressively
        measurable. This implies that the optional $\sigma$-algebra $\cO$ and 
        the predictable $\sigma$-algebra $\cP$ are contained in the 
        progressive $\sigma$-algebra $\cR$. 

    \item Conversely, a measurable and adapted process is progressively
        measurable.

    \item $\cP \subset \cO$. 

    \item $\cR \subset\cM$, where $\cM$ is the measurable $\sigma$-algebra on
        $\Omega\times \R_{+}$. 
\end{enumerate}

\problem{Filtrations and continuity of paths.} Let $X$ be a stochastic process
on a probability space $(\Omega, \cA, P)$ and $\cF^{X}=(\cF^{X}_{t})_{t\geq 0}$
be the natural filtration of $X$. Moreover, $A$ is the event that $X$ is
continuous on $[0, t_0)$. Then following holds:
\begin{enumerate}
    \item If $X$ has c\`adl\`ag paths, i.e. every path of $X$ is a c\`adl\`ag
        function, then $A\in \cF^{X}_{t_0}$. 

    \item If $X$ is c\`adl\`ag, i.e. $X$ has c\`adl\`ag paths a.s., then $A$
        may fail to be in $\cF^{X}_{t_0}$.

    \item If $X$ is c\`adl\`ag, and $(\cF_{t})_{t\geq 0}$ is a filtration
        containing $\cF^{X}$ such that $\cF_{0}$ contains all $P$-null events
        in $\cA$, then $A\in \cF_{t_0}$.

    \item If $X$ has c\`agl\`ad paths and is adapted to a right-continuous
        filtration $(\cF_{t})_{t\geq 0}$, then $A\in \cF_{t_{0}}$. 
\end{enumerate}


\section{Stopping times}

\problem{Stopping times. Basic properties.} Assume that the stochastic basis
$(\Omega, \cA, \bF, P)$ satisfies the usual conditions, and $T$, $S$, and $S_n,
n\in\bN$ are stopping times. Then the following holds.
\begin{enumerate}
    \item $T + t_0$ for $t_0\geq 0$ is a stopping time.
    \item $T+S$ is a stopping time.
    \item A fixed time $U \equiv t_0>0$ is a stopping time and we have $\cF_U = \cF_{t_0}$.
    \item $\bigvee_{n\in\bN} S_n$ and $\bigwedge_{n\in\bN} S_n$ are stopping times. 
\end{enumerate}

\solution
\begin{enumerate}
    \item \begin{align*}
            \cF_T &=& 
            \left\{ A\in\bF : A \cap \left\{ t_0 \leq t \right\}\in\cF_t \ \forall t\geq 0 \right\} \\
            &=& \left\{ A\in\bF : A\in\cF_t \ \forall t\geq t_0 \right\} = \cF_{t_0}.
        \end{align*}
\end{enumerate}

\problem{Stopping time $\sigma$-algebras.} Let $T$ be a stopping time on a
stochastic basis $(\Omega, \cA, \bF, P)$.
\begin{enumerate}
    \item $\cF_T$ is a $\sigma$-algebra.
    \item $\cF_{T-} \subset \cF_{T}$.
    \item $T$ is $\cF_{T-}$-measurable.
\end{enumerate}

\problem{Stopping times and continuity of filtrations.}
Consider a filtered probability space $(\Omega, \cA, \bF, P)$ with $\Omega =
[0, \infty)$ and $\cF_t = \cB([0,t]) \cup \left\{ (t,\infty), [0, \infty)
\right\}$ for every $t\geq 0$. We define a random time $T:\Omega\to[0,\infty),
\omega \mapsto \omega$.  Then the following holds
\begin{enumerate}
    \item $\cF_{t} = \cF_{t-} = \cF_{t+}$.
    \item $\cF_{T-} = \cF_{T}$.
\end{enumerate}
Is $T$ a stopping time?

\problem{Stopping times. I}
On a stochastic basis $(\Omega, \cA, \bF, P)$ define a stopping time $T$ and a
real random variable $Z$. If $T \leq Z$ and $Z$ is $\cF_{T}$ measurable, then
$Z$ is a stopping time.

\problem{Stopping times and right-continuous filtrations.} Let $(\Omega, \cA,
\bF, P)$ be a filtered probability space. A random time $T$ satisfying
$\{T<t\}\in\cF_{t}$ for all $t\geq 0$ is called an optional time. Prove the
following statements.
\begin{enumerate}
    \item Assume that $\bF$ is right-continuous. $T$ is a stopping time, iff it
        is an optional time.
    \item Show that the right-continuity of $\bF$ is essential for the above
        property to hold: Construct an optional time which is not a stopping
        time.
\end{enumerate}

\solution  Betrachte die Darstellung
\begin{eqnarray}
    \left\{ T \leq t \right\} = \bigcap_{\varepsilon\in\bQ^{+}} \left\{ T > t+\varepsilon \right\}
    = \lim_{\varepsilon\to 0, \varepsilon\in\bQ} \left\{ T > t+\varepsilon \right\}.
\end{eqnarray}
Der Schnitt ist als Mengengrenzwert zu verstehen, weil die Mengen $\left\{ T >
t+\varepsilon \right\}$ ineinandergeschachtelt sind. $\left\{ T \leq t \right\}\in \bF_{t+}$
genau dann, wenn $\left\{ T \leq t \right\}\in \bF_{t+\varepsilon^{*}}$ $\forall \varepsilon^{*}>0$.
Es gilt aber für ein fixes $\varepsilon^{*}$
\begin{equation}
    \left\{ T \leq t \right\}\ = 
    \bigcap_{\varepsilon\in\bQ^{+}, \varepsilon\leq \varepsilon^{*}} \left\{ T > t+\varepsilon \right\} 
    \in\bF_{t+\varepsilon^{*}},
\end{equation}
denn es ein abzählbarer Schnitt der Mengen $\left\{ T> t+\varepsilon
\right\}\in\bF_{t+\varepsilon^{*}}$ ist. Behauptung gilt wegen $\cF_t=\cF_{t+}$ $\forall t$.

\problem{Stopping times. Counterexample.} Give an example of stopping times $S$
and $T$, with $S\leq T$ for which $T-S$ is not a stopping time. 

\problem{Sequences of stopping times.} 
$(T_n)_{n\in\bN}$ is a sequence of stopping times decreasing to a random time
$T$. Show that $T$ is a stopping time, and that $\cF_{T} = \bigcap_{n\in\bN}
\cF_{T_n}$. 



